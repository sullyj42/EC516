\documentclass{article}
\usepackage{graphicx}
\usepackage{rotating}
\usepackage{float}
\usepackage{epstopdf}
\usepackage[T1]{fontenc}
\usepackage[utf8]{inputenc}
\usepackage[margin=1in]{geometry}
\usepackage{amsmath}
\newcommand{\HRule}{\rule{\linewidth}{0.5mm}}
\newcommand{\Hrule}{\rule{\linewidth}{0.3mm}}
\usepackage{amssymb}
\newcommand{\volume}{\mathop{\ooalign{\hfil$V$\hfil\cr\kern0.08em--\hfil\cr}}\nolimits}
\usepackage{ mathrsfs }
\newcommand*{\QEDA}{\hfill\ensuremath{\blacksquare}}%
\newcommand*{\QEDB}{\hfill\ensuremath{\square}}%
\renewcommand\thesection{Problem \arabic{section}: \hspace*{-1em}}
\renewcommand\thesubsection{\alph{subsection}.}
%\newcommand{\pder}[2][]{\frac{\partial#1}{\partial#2}}
\usepackage{bm}
\usepackage{multicol}
\usepackage{paracol}
\usepackage{mdframed}
\usepackage{mhchem}
\newcommand{\putpicture}[4]{
	\begin{figure}[H]
		\centering
		\includegraphics[width = #2\textwidth]{#1}
		\caption{#3}
		\label{#4}
	\end{figure}
}
%\usepackage[style=apa]{biblatex}
%\addbibresource{apabib.bib}
% \usepackage{../../latexStyles/mcode}
\usepackage[natbibapa]{apacite}
\setlength\bibsep{\baselineskip}
\usepackage{fancyhdr}
\def\changemargin#1#2{\list{}{\rightmargin#2\leftmargin#1}\item[]}
\let\endchangemargin=\endlist 
\makeatletter% since there's an at-sign (@) in the command name
\renewcommand{\@maketitle}{%
	\parindent=0pt% don't indent paragraphs in the title block
	\centering
	{\Large \bfseries\textsc{\@title}} \hfill \textit{Digital Signal Processing} \hfill {\Large EC516}
	\HRule\par%
	\textit{\@author \hfill \hspace*{.3in} \hfill \@date \\
	\hfill \hspace*{.3in} \hfill}
	\par
}
\makeatother% resets the meaning of the at-sign (@)
\usepackage{../../mcode}
\DeclareMathOperator{\arcsinh}{arcsinh}

\title{Homework 2}
\author{Jeremiah Sullivan}
% \program{Ocean Acoustics}
\date{\today}

\pagestyle{fancy}
\lhead{Jeremiah Sullivan}
\rhead{\today}
\chead{HW2}
\cfoot{Page \thepage}
\lfoot{Spring 2017}
\rfoot{EC516}
\renewcommand{\footrulewidth}{0.4pt}% default is 0pt
\usepackage{blindtext}
\usepackage{mwe}
\begin{document}
\maketitle
\section{Real Valued Signal}
Suppose	 $x(t)$ is	a	 real-valued	speech-signal	whose	Fourier	 transform	is	 $X ( j\omega)$ and	it	is	 known	 that	$| X ( j\omega)|= 0$ for	$|\omega | \geq 10,000\pi|$ .	 Let	 $x[n] = x(nT )$ where	 T	 represents	the	 sampling	 interval.	 Answer	 the	 following	 questions	 about	 $X (e^{j\omega}) )$ ,	 the	 DTFT	 of	 x[n],	for	the	specified	values	of	T.\\[0.1in]
First, since the signal is real-valued, I belive the FT should be symmetric about zero (no information in negatives). Though this does not matter for the problem. \\[0.1in]
The following questions all essentially ask where the frequency content is definitely zero, given the band-limiting conditions above. Note that, when sampled, the DFT repeats every $2\pi/T$. The response then becomes "compressed" in frequency. 
\putpicture{prob1}{0.5}{Basic Signal}{exactRate}
\subsection{Zeros}
\subsubsection{T = 0.0001 seconds}	
With this sample rate, we have replication occur at $2\pi / 1E-4$. Setting this symmetric about zero, we have a signal repeating on the interval [$-\pi E4, \pi E4].$ This is exactly equal to our band-limiting condition. Thus the only guaranteed zeros occur at $\pi E4*(2k - 1), k \in \mathbb{Z}$. 
\putpicture{prob1a}{0.5}{Repetition}{exactRate}
\subsubsection{T = T1/2}

Similar to problem 1, but our band now becomes limited at $2\pi / 5E-5,$ or four times the band limiting rate. Effectively doubling the bandwidth from the last question. The bandwidth is zero at the specified value, 10,000$\pi$. The function folds at $2\pi E4*(2k - 1)$. Thus we have zeros in all regions, [10,000$\pi k$, $2\pi E4*(2k - 1) + 10,000\pi$.
\putpicture{prob1b}{0.5}{Repetition}{exactRate}
\subsubsection{T1/10}
Using the same process as above, we have zeros between [10,000$\pi k$, $2\pi E4*(2k - 1) + 10,000\pi$.
\putpicture{prob1c}{0.5}{Repetition}{exactRate}
\subsection{Magnitude of Fourier Transform}
\begin{align*}
 \delta[n-3]  &\to |e^{3j\omega}| = 1.\\
\delta[n-3]  + \delta[n+3] &\to |e^{3j\omega} e^{-3j\omega}| = 2\cos(3\omega).\\
u[n] - u[n-4] \text{(five sample box) } &\to \frac{\sin(4\omega/2)}{ \sin(\omega/2))}e^{-j\omega \cdot 3/2} 
\end{align*}
\putpicture{prob2}{0.5}{DFT Magnitudes}{mags}
\subsection{Phase of DTFT}
First, construct the full signal. Note the convolution of a box with a box should be a triangle (potentially with its head cut off). $\Sigma  conv(x[n], x[n-4]) = \Sigma ((u[n] - u[n-4])e^{-j\omega})$

\end{document}